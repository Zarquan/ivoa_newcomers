% The following are the VO standards sometimes explained from into the
% tutorials. We proved shortcuts for these standard texts here (similar
% like for the vo tools).

\newenvironment{WGinfo}[1]{ % Infos on IVOA Working Groups, IGs and commitees.
	% use \s{} to highlight individual WGs.
  \def\s##1{{\color{VOkey}##1}}
  
{\subsubsection*{\color{VOnum}#1}}
}


\newcommand{\ADQLSTD}{
\lstset{style=VO}
\begin{standardsinfo}{ADQL}
\s{ADQL} (Astronomical Data Query Language) is a query language
based on a subset of SQL used to query tabular data in TAP services.
The \s{ADQL} standard defines a common subset of SQL
that can be used to access data in all the common relational
database platforms used in astronomy.
This enables users to query multiple services
using the same query without having to worry about the
specific dialect of the relational database platform
hosting the data.
\end{standardsinfo}
}

\newcommand{\TAPSTD}{
\lstset{style=VO}
\begin{standardsinfo}{TAP}
\s{TAP} (Table Access Protocol)
defines a standard web service interface
for querying tabular catalogs.
The protocol defines how to represent the query parameters
and the available response formats.
This enables users to query multiple services
using the same client software and query
language.
\end{standardsinfo}
}

\newcommand{\REGISTRYSTD}{
\lstset{style=VO}
\begin{standardsinfo}{VO Registry}
The \s{Registry} is the central point for searches for services in the VO.
VO services identify themselves to the Registry, providing metadata about
the data they serve and the service protocols they support.
Thus, the Registry is the entry point for data discovery in the VO.
\end{standardsinfo}
}

\newcommand{\UCD}{
\lstset{style=VO}
\begin{standardsinfo}{UCDs -- VO semantics}
\s{UCD}s (Unified Content Descriptors) describe the content of
table columnms in a machine readable way.
Thus, the machines ``know'' something about the data and you
can make use of this in your programs.
The data type of a column may describe it as a floating point number,
the units may say it is measured in degrees,
but the UCD allows us to identify it is a
\textit{right ascension position coordinate},
enabling data processing and display software
to understand how to use it.
\end{standardsinfo}
}

\newcommand{\VOTABLE}{
\lstset{style=VO}
\begin{standardsinfo}{VOTABLE}
\s{VOTable} is the standard data exchange format for tabular data in the VO.
The \s{VOTable} standard defines an XML schema
that combines data types, units,
UCDs and data model references
to provide machine readable metadata describing
each column in the table,
enabling data processing and display software
to understand how to use it.
\end{standardsinfo}
}

\newcommand{\SAMP}{
\lstset{style=VO}
\begin{standardsinfo}{SAMP}
\s{SAMP} (Simple Application Messaging Protocol)
enables astronomy applications on the same desktop
or laptop machine to share data with each other.
It also enables applications to notify each other
about what data points a user has selected,
enabling the two applications to coordinate their
display and selection views.
\end{standardsinfo}
}

\newcommand{\SIAP}{
\lstset{style=VO}
\begin{standardsinfo}{SIAP}
\s{SIAP}SIAP  (Simple Image Access Protocol)
defines a standard web service interface
for finding and accessing images
in the vO.
The protocol defines the query parameters
the service understands and the format of the response.
This enables users to query multiple services
using the same criteria.
Each service returns a machine readable VOTable document
listing the
available images
that match the search criteria
and describes how to access them.
\end{standardsinfo}
}

\newcommand{\SSAP}{
\lstset{style=VO}
\begin{standardsinfo}{SSAP}
\s{SSAP} (Simple Spectra Access Protocol)
defines a standard web service interface
for finding and accessing spectral
data in the vO.
The protocol defines the query parameters
the service understands and the format of the response.
This enables users to query multiple services
using the same criteria.
Each service returns a machine readable VOTable document
listing the
available spectra
that match the search criteria
and describes how to access them.
\end{standardsinfo}
}

\newcommand{\OBSLOCTAP}{
\lstset{style=VO}
\begin{standardsinfo}{ObsLocTAP}
\s{ObsLocTAP} (Observation Locator Table Access Protocol)
defines a standard data model for querying
observation scheduling information
for an instrument.
The protocol defines the query parameters
the service understands and the format of the response.
This enables users to query multiple services
using the same criteria.
Each service returns a machine readable VOTable document
listing the
planned or completed observations
that match the search criteria.
\end{standardsinfo}
}

\newcommand{\OBSVISSAP}{
\lstset{style=VO}
\begin{standardsinfo}{ObjVisSAP}
\s{ObjVisSAP} (Object Visibility Simple Access Protocol)
defines a standard web service interface
for discovering visibility time intervals
for an instrument.
The protocol defines the query parameters
the service understands and the format of the response.
This enables users to query multiple services
using the same criteria.
Each service returns a machine readable VOTable document
listing the
visibility time intervals
that match the search criteria.
\end{standardsinfo}
}

\newcommand{\OBSCORE}{
\lstset{style=VO}
\begin{standardsinfo}{ObsCore}
\s{ObsCore}
Is a common data model for describing astronomical observations.
It defines the core components of metadata needed to discover what observational data is available from a service.
Every service that advertises itself as an \s{ObsTAP} service must provide this standard view of
metadata lsiting the observational data available from that service..
This means that a user can build an \s{ADQL} query based on the \s{ObsCore} data table
and apply the same query to all the \s{ObsTAP} services.
\end{standardsinfo}
}

\newcommand{\SCS}{
\lstset{style=VO}
\begin{standardsinfo}{SCS}
\s{SCS} (Simple Cone Search)
defines a standard web service interface
for requesting data based on a cone projected on the sky
described by a sky position and an angular distance.
The protocol defines the query parameters
the service understands and the format of the response.
This enables users to query multiple services
using the same criteria.
Each service returns a machine readable VOTable document
listing the astronomical sources or objects which are
within the search criteria.
\end{standardsinfo}
}

\newcommand{\DATALINK}{
\lstset{style=VO}
\begin{standardsinfo}{Datalink}
\s The {Datalink} standard connects meta data of datasets to the
data, related data products, and web service capabilities that can act
upon the data, without having to download the whole dataset.  
\end{standardsinfo}
}




% Working Groups


\newcommand{\APPS}{
\lstset{style=VO}
\begin{WGinfo}{Application WG}
As the title \s{Applications WG 1} implies, the focus of this WG is on
the perspective on the VO from the view of an application
developer. Thus, the Sessions include discussions and exchange
of VO standards implementation, but also new demands to these
standards deriving from the use of applications. The application WG
could be linked to any standard or protocol an application might speak
or consume, so the involvement is everywhere in the VO. You will find
further information here:
\newline
\url{https://wiki.ivoa.net/twiki/bin/view/IVOA/IvoaApplications}
\end{WGinfo}
}

\newcommand{\DAL}{
\lstset{style=VO}
\begin{WGinfo}{Data Access Layer WG}
The \s{Data Access Layer WG} is tasked with the definition and
formalizatio of standards for remote data access. As such, the
WG is in charge of standards like SCS, SIAP, TAP/ADQL and
others. A more detailed description can be found here:
\newline
\url{https://wiki.ivoa.net/twiki/bin/view/IVOA/IvoaDAL}
\end{WGinfo}
}

\newcommand{\DATAMODEL}{
\lstset{style=VO}
\begin{WGinfo}{Data Model WG}
The role of the \s{Data Modeling Working Group} is to provide a framework for the description of metadata attached to observed or simulated data. 
\newline
\url{https://wiki.ivoa.net/twiki/bin/view/IVOA/IvoaDataModel}
\end{WGinfo}
}

\newcommand{\GWS}{
\lstset{style=VO}
\begin{WGinfo}{Grid and Web Services WG}
The aim of the \s{GWS WG} is to define the use of Grid technologies and web services within the VO context and to investigate, specify, and implement required standards in this area. 
\newline
\url{https://wiki.ivoa.net/twiki/bin/view/IVOA/IvoaGridAndWebServices}
\end{WGinfo}
}

\newcommand{\REGISTRY}{
\lstset{style=VO}
\begin{WGinfo}{Resource Registry WG}
The \s{Resource Registry WG} is responsible for maintaining the live
registry and extending the metadata to describe new service. 
\newline
\url{https://wiki.ivoa.net/twiki/bin/view/IVOA/IvoaResReg}
\end{WGinfo}
}

\newcommand{\SEMANTICS}{
\lstset{style=VO}
\begin{WGinfo}{Semantics WG}
The \s{Semantics Working Group} (Charter) superseded the UCD Working Group in 2006 as it was realised semantic resources beyond UCDs were required. To this day, the main focus of the Semantics WG is the maintenance and development of Vocabularies. 
\newline
\url{https://wiki.ivoa.net/twiki/bin/view/IVOA/IvoaSemantics}
\end{WGinfo}
}


% Interest Groups
\newcommand{\DCP}{
\lstset{style=VO}
\begin{WGinfo}{Data Curation and Preservation IG}
\s{Data Curation and Preservation IG}
\newline
\url{https://wiki.ivoa.net/twiki/bin/view/IVOA/IvoaDCP}
\end{WGinfo}
}

\newcommand{\EDU}{
\lstset{style=VO}
\begin{WGinfo}{Education IG}
\s{Education}
\newline
\url{ https://wiki.ivoa.net/twiki/bin/view/IVOA/IvoaEducation }
\end{WGinfo}
}

\newcommand{\KDD}{
\lstset{style=VO}
\begin{WGinfo}{Knowledge Discovery IG }
\s{Knowledge Discovery IG}
\newline
\url{ https://wiki.ivoa.net/twiki/bin/view/IVOA/IvoaKDD }
\end{WGinfo}
}

\newcommand{\OPERATIONS}{
\lstset{style=VO}
\begin{WGinfo}{Operations IG}
\s{Operations IG}
\newline
\url{ https://wiki.ivoa.net/twiki/bin/view/IVOA/IvoaOps }
\end{WGinfo}
}

\newcommand{\RADIO}{
\lstset{style=VO}
\begin{WGinfo}{Radio astronomy IG}
\s{Radio astronomy IG}
\newline
\url{https://wiki.ivoa.net/twiki/bin/view/IVOA/IvoaRadio }
\end{WGinfo}
}

\newcommand{\SOLAR}{
\lstset{style=VO}
\begin{WGinfo}{Solar System IG}
\s{Solar System IG}
\newline
\url{ https://wiki.ivoa.net/twiki/bin/view/IVOA/IvoaSS }
\end{WGinfo}
}



\newcommand{\THEORY}{
\lstset{style=VO}
\begin{WGinfo}{Theory IG}
\s{Theory IG}
\newline
\url{https://wiki.ivoa.net/twiki/bin/view/IVOA/IvoaTheory }
\end{WGinfo}
}

\newcommand{\TIMEDOMAIN}{
\lstset{style=VO}
\begin{WGinfo}{Time Domain IG}
\s{Time Domain IG}
\newline
\url{https://wiki.ivoa.net/twiki/bin/view/IVOA/IvoaVOEvent}
\end{WGinfo}
}


% Exec, TCG SCSP, SCSP
\newcommand{\EXEC}{
\lstset{style=VO}
\begin{WGinfo}{Exec}
\s{IVOA Executive}
\newline
\url{https://wiki.ivoa.net/twiki/bin/view/IVOA/IvoaRepMin}
\end{WGinfo}
}

\newcommand{\TCG}{
\lstset{style=VO}
\begin{WGinfo}{TCG}
\s{IVOA Technical Coordination Group}
\newline
\url{https://wiki.ivoa.net/twiki/bin/view/IVOA/IvoaTCG}
\end{WGinfo}
}

\newcommand{\CSP}{
\lstset{style=VO}
\begin{WGinfo}{CSP}
\s{IVOA Standing Committee on Science Priorities}
\newline
\url{https://wiki.ivoa.net/twiki/bin/view/IVOA/IvoaSciencePriorities}
\end{WGinfo}
}

\newcommand{\SCSP}{
\lstset{style=VO}
\begin{WGinfo}{SCSP}
\s{Standing Committee on Standards and Processes }
\newline
\url{https://wiki.ivoa.net/twiki/bin/view/IVOA/IvoaSciencePriorities}
\end{WGinfo}
}

\newcommand{\glossitem}[2]
{ \item \color{VOnum}{{#1}} : \color{black} {{#2}}
}


% https://wiki.ivoa.net/twiki/bin/view/IVOA/VOGlossary


%\newcommand{}{
%\lstset{style=VO}
%\begin{standardsinfo}{}
%\s{}
%
%
%\end{standardsinfo}
%}

